\documentclass[Harvard,Times1COL]{WileyNJDv5}


\usepackage{natbib}
\bibliographystyle{wileyNJD-Harvard}

\hypersetup{
  colorlinks=true,
  linkcolor=blue,
  citecolor=blue,
  filecolor=black,
  urlcolor=black
}
\begin{document}

\articletype{Article Type: Research Article}

\received{Date Month Year}
\revised{Date Month Year}
\accepted{Date Month Year}
\journal{Journal of Policy Analysis and Management}
\volume{00}
\copyyear{2025}
\startpage{1}

% TODO: Inherited from WileyNJD example. Is this needed?
% \raggedbottom


\authormark{Dupré and Morgenroth}

\presentaddress{DCU Business School, Dublin 9, Ireland.}



% Include information for each author
  \author[1]{Damien Dupré}
  \author[1]{Edgar Morgenroth}

% Include information for each affiliation
\address[1]{%
  \orgdiv{%
    Business School%
    , %
    \orgname{%
      Dublin City University%
      , %
    }%
    \orgaddress{%
      \state{Dublin}%
      , %
      \country{Ireland}%
    }%
  }%
}

% Include information for each corresponder
      \corres{%
      %
        Damien Dupré%
      %
      \email{damien.dupre@dcu.ie}%
    }
    
\title{Divergent Impacts of COVID-19 School Closures on Youth Infection
Dynamics: Evidence from 12 European Countries}
\titlemark{COVID-19 School Closures and Youth Infection Dynamics}

\abstract[Abstract]{The effectiveness of school closures as a COVID-19
control measure remains a subject of debate. This paper critically
reassesses their impact by examining infection trends across various age
groups in 12 European countries. We apply a dual analytical approach:
Generalised Additive Models (GAMs) are used to capture complex,
non-linear effects of closures on age-specific case numbers, while
Transfer Entropy (TE) is employed to quantify directional transmission
patterns between these age groups. Our findings reveal deviations from
commonly held assumptions. While school closures were associated with a
non-linear decline in overall national COVID-19 cases, the effects on
children and young adults varied considerably. A consistent downward
trend in infections was seen only in the pre-school age group. In
contrast, school-aged children experienced a marked rise in COVID-19
cases following an initial period of stability or slight decrease after
closures began. The Transfer Entropy analysis also identified
asymmetric, directional patterns in transmission, revealing that
infection trends in specific age groups could predict subsequent changes
in others. These results challenge the assumption that school closures
uniformly benefit younger populations and emphasise the need for
age-specific analysis in pandemic planning and support the development
of more nuanced, evidence-based public health policies.}

{\keywords{COVID-19, School closures, Intergenerational
transmission, Non-linear effects, Infection dynamics.}

\maketitle


\end{document}
